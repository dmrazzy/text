% Copyright 2004 by Till Tantau <tantau@users.sourceforge.net>.
%
% In principle, this file can be redistributed and/or modified under
% the terms of the GNU Public License, version 2.
%
% However, this file is supposed to be a template to be modified
% for your own needs. For this reason, if you use this file as a
% template and not specifically distribute it as part of a another
% package/program, I grant the extra permission to freely copy and
% modify this file as you see fit and even to delete this copyright
% notice. 

\documentclass{beamer}

\usepackage[utf8]{inputenc}
\usepackage{listings}

\newcommand{\code}[1]{\texttt{#1}}

\lstset{ %
language=C++,           % choose the language of the code
numbers=left,           % where to put the line-numbers
numberstyle=\tiny,      % the size of the fonts that are used for the line-numbers
basicstyle=\footnotesize    % the size of the fonts that are used for the line-numbers
}

% There are many different themes available for Beamer. A comprehensive
% list with examples is given here:
% http://deic.uab.es/~iblanes/beamer_gallery/index_by_theme.html
% You can uncomment the themes below if you would like to use a different
% one:
%\usetheme{AnnArbor}
%\usetheme{Antibes}
%\usetheme{Bergen}
%\usetheme{Berkeley}
\usetheme{Berlin}
%\usetheme{Boadilla}
%\usetheme{boxes}
%\usetheme{CambridgeUS}
%\usetheme{Copenhagen}
%\usetheme{Darmstadt}
%\usetheme{default}
%\usetheme{Frankfurt}
%\usetheme{Goettingen}
%\usetheme{Hannover}
%\usetheme{Ilmenau}
%\usetheme{JuanLesPins}
%\usetheme{Luebeck}
%\usetheme{Madrid}
%\usetheme{Malmoe}
%\usetheme{Marburg}
%\usetheme{Montpellier}
%\usetheme{PaloAlto}
%\usetheme{Pittsburgh}
%\usetheme{Rochester}
%\usetheme{Singapore}
%\usetheme{Szeged}
%\usetheme{Warsaw}

\title{SMT-based Compile-time Verification of Safety Properties for Smart Contracts}
%\subtitle{Optional Subtitle}

\author{Leonardo Alt \and Christian Reitwiessner}

\institute{Ethereum Foundation}

% Delete this, if you do not want the table of contents to pop up at
% the beginning of each subsection:
\AtBeginSection[]
{
  \begin{frame}<beamer>{Outline}
    \tableofcontents[currentsection,currentsubsection]
  \end{frame}
}

% Let's get started
\begin{document}

\begin{frame}
  \titlepage
\end{frame}

\begin{frame}{Outline}
  \tableofcontents
\end{frame}

\section{Ethereum}

\begin{frame}{Ethereum}
\begin{itemize}
	\item Global networked application platform \bigskip
	\item Distributed public database \bigskip
	\item Blockchain consensus \bigskip
	\pause
	\item Trustless \bigskip
	\item Transparent \bigskip
	\item No single entity has control \bigskip
\end{itemize}
\end{frame}

\begin{frame}{Smart Contracts}
\begin{itemize}
	\item Accounts controlled by code \bigskip
	\item SCs have a storage which can only be written by the SC itself \bigskip
	\item The code runs on the Ethereum Virtual Machine (EVM) \bigskip
	\item Opcodes have costs (\emph{gas}) \bigskip
	\item Transactions may \emph{revert} (exception termination state) \bigskip
\end{itemize}
\end{frame}

\begin{frame}{Smart Contracts}
\begin{itemize}
	\item The code is public and immutable after creation \bigskip
	\item Monetary incentive to study/attack the program \bigskip
	\item The DAO and Parity hacks/accidents \bigskip
\end{itemize}
\pause
Easy, just use formal verification :)
\end{frame}

\begin{frame}{Smart Contracts - Verification Frameworks}
\begin{itemize}
	\item EVM Formal Semantics: Eth-Isabelle~\cite{Eth-Isabelle}, KEVM~\cite{KEVM}, Ethereum-Lem~\cite{Ethereum-Lem} \bigskip
	\item EVM bytecode Symbolic Execution: Oyente~\cite{Luu2016}, Mythril~\cite{Mythril}, MAIAN~\cite{NKSSH18} \bigskip
	\item Translation of Solidity to verifiable languages: Why3~\cite{Why3}, F*~\cite{Bhargavan2016}, ZEUS~\cite{ZEUS} \bigskip
	\item Our approach: Solidity SMT-based BMC
	\begin{itemize}
		\item Part of the compilation stack
		\item No extra effort spent by the developer
		\item Counterexamples
	\end{itemize}
\end{itemize}
\end{frame}


%\begin{frame}{Ethereum}
%Use cases of smart contracts:
%\begin{itemize}
%	\item Decentralized Tokens \bigskip
%	\item Exchanges, Auctions \bigskip
%	\item Notary, Registry \bigskip
%	\item Insurance, Lending \bigskip
%\end{itemize}
%\end{frame}

\section{Solidity BMC}

\begin{frame}{Solidity}
\begin{itemize}
	\item Developed for smart contracts \bigskip
	\item Most used language to write smart contracts \bigskip
	\item Syntax similar to C/Java \bigskip
	\item Main code elements are \emph{contracts} \bigskip
	\item Storage and local variables \bigskip
	\item Integers of various sizes, \emph{address}, \emph{mapping}, structs, arrays...
\end{itemize}
\end{frame}

\begin{frame}[fragile]{Solidity}
\begin{lstlisting}[
    basicstyle=\tiny, %or \small or \footnotesize etc.
]
contract Token {
    /// The main balances / accounting mapping.
    mapping(address => uint256) balances;

    /// Create the token contract crediting `msg.sender` with
    /// 10000 tokens.
    constructor() public {
        balances[msg.sender] = 10000;
    }

    /// Transfer `_value` tokens from `msg.sender` to `_to`.
    function transfer(address _to, uint256 _value) public {
        require(balances[msg.sender] >= _value);
        uint256 sumBefore = balances[msg.sender] + balances[_to];
        balances[msg.sender] -= _value;
        balances[_to] += _value;
        uint256 sumAfter = balances[msg.sender] + balances[_to];
        assert(sumBefore == sumAfter);
    }
}
\end{lstlisting}
\end{frame}

\begin{frame}{Solidity BMC}
\begin{itemize}
	\item Basic idea: SMT-based BMC \bigskip
	\item Gas incentive against unbounded loops \bigskip
	\item Practicality over Completeness \bigskip
	\item Precise encoding
\end{itemize}
\end{frame}

\begin{frame}{Solidity BMC}
Verification targets:\\
\begin{itemize}
	\item Underflow / Overflow / Division by 0 \bigskip
	\item Trivial conditions / Unreachable code \bigskip
	\item Assertions
\end{itemize}
\end{frame}

\begin{frame}{Solidity $\rightarrow$ SMT Encoding}
\begin{itemize}
	\item Traverse the AST
	\begin{itemize}
		\item Collect constraints
		\item Query the SMT solver \bigskip
	\end{itemize}
	\item Five types of constraints:\\Control-flow, Type constraint, Variable assignment, Branch conditions, Verification Target
\end{itemize}
\end{frame}

\begin{frame}{Solidity $\rightarrow$ SMT Encoding}
Branch conditions:\\~\\
Auxiliary stack that keeps track of the conjunction of conditions that are
true at the current program path.\\
No constraint is added to the solver.
\end{frame}

\begin{frame}{Solidity $\rightarrow$ SMT Encoding}
Control-flow:\\~\\
Global constraints of the form $b \rightarrow r$, where $b$
is the conjunction of conditions in the current path and $r$ is
the condition in \code{require(r)} and \code{assert(r)}.
\end{frame}

\begin{frame}{Solidity $\rightarrow$ SMT Encoding}
Type constraint:\\~\\
Declaration of local variables use default values from types
(Integer: 0, Bool: false), and function parameters are initialized
with a range of valid values (\code{uint32 x}: $0 \le x < 2^{32}$).
\end{frame}

\begin{frame}{Solidity $\rightarrow$ SMT Encoding}
Variable assignment:\\~\\
The encoding follows the SSA form.
When a variable is assigned inside different branches,
a new \code{if-then-else} variable is created after the
branching to re-combine the different values.
\end{frame}

\begin{frame}{Solidity $\rightarrow$ SMT Encoding}
Verification target:\\
\begin{itemize}
	\item Underflow / Overflow / Division by 0 \bigskip
	\item Trivial conditions / Unreachable code \bigskip
	\item Assertions \bigskip
\end{itemize}
Counterexamples are given when a property fails.
\end{frame}

\begin{frame}[fragile]{Solidity $\rightarrow$ SMT Encoding}
\begin{figure}
\noindent\begin{minipage}{.48\textwidth}
\begin{lstlisting}[
    basicstyle=\tiny, %or \small or \footnotesize etc.
]
contract C
{
  function f(uint256 a, uint256 b)
  {
    if (a == 0)
      require(b <= 100);
    else if (a == 1)
      b = 1000;
    else
      b = 10000;
    assert(b <= 100000);
  }
}
\end{lstlisting}
\end{minipage}\hfill
\begin{minipage}{.48\textwidth}
1. $a_0 \ge 0 \land a_0 < 2^{256}  \land \phantom{x}$\\
2. $b_0 \ge 0 \land b_0 < 2^{256}  \land \phantom{x}$\\
3. $(a_0 = 0) \rightarrow (b_0 \le 100) \, \land$\\
4. $b_1 = 1000 \land b_2 = 10000$\\
5. $b_3 = ite(a == 1, b_1, b_2) \land \phantom{x}$\\
6. $b_4 = ite(a == 0, b_0, b_3) \land \phantom{x}$\\
7. $\neg b_4 \le 100000$
\end{minipage}
\end{figure}

\end{frame}

\section{Future Plans}

\begin{frame}[fragile]{Future Plans - Loops}
\begin{itemize}
	\item Automatic detection of loop bounds \bigskip
	\item Invariant annotations \bigskip
	\item Current syntax:
\end{itemize}
\begin{lstlisting}[
    basicstyle=\footnotesize, %or \small or \footnotesize etc.
]
contract C
{
    uint x;
    function f(uint[] _arr) {
        require(_arr.length < 12);
        for (uint i = 0; i < _arr.length; ++i)
            x += _arr[i];
    }
}
\end{lstlisting}
\end{frame}

\begin{frame}[fragile]{Future Plans - Multi-transaction Invariants}
\begin{lstlisting}[
    basicstyle=\tiny, %or \small or \footnotesize etc.
]
contract C
{
  uint a;

  constructor () public {}

  function a1() public { a = 1; }
  function a2() public { a = 2; }
  function a3() public { a = 3; }
  function a4() public { a = 4; }

  function plusA(uint x) public view returns (uint) {
    require(x < 1000);
    return a + x;
  }
}
\end{lstlisting}
No transaction leads to $a > 4$
\end{frame}

\begin{frame}[fragile]{Future Plans - Post-constructor Invariants}
\begin{lstlisting}[
    basicstyle=\tiny, %or \small or \footnotesize etc.
]
contract C
{
  uint a;

  constructor (uint x) public {
	  require(x < 10);
	  a = x;
  }

  function plusA(uint x) public view returns (uint) {
    require(x < 1000);
    return a + x;
  }
}
\end{lstlisting}
State variable $a$ can only be assigned values smaller than 10,
and is never assigned again.
\end{frame}

\begin{frame}[fragile]{Future Plans - Modifiers as pre and postconditions}
\begin{lstlisting}[
    basicstyle=\tiny, %or \small or \footnotesize etc.
]
contract C
{
    uint totalSupply;

    modifier safeBalance {
        require(totalSupply == 10000);
        _;
        assert(totalSupply == 10000);
    }

    function transfer(address _to, uint256 _value) safeBalance {
        ...
    }
}
\end{lstlisting}
Modifier \code{safeBalance} guarantees that the \code{totalSupply}
does not change after a \code{transfer}.
\end{frame}

\begin{frame}[fragile]{Future Plans - Function abstraction}
\begin{lstlisting}[
    basicstyle=\tiny, %or \small or \footnotesize etc.
]
contract C
{
    uint totalSupply;

    modifier safeBalance {
        require(totalSupply == 10000);
        _;
        assert(totalSupply == 10000);
    }

    function transfer(address _to, uint256 _value) safeBalance {
        ...
    }

    function zeroAccount(address _to) {
        transfer(_to, balance[msg.sender]);
        assert(totalSupply == 10000);
    }
}
\end{lstlisting}
The modifier can then be used to abstract function calls and prove
properties more efficiently.
\end{frame}

\begin{frame}{Future Plans - Range restriction of real life values}
\begin{itemize}
	\item \code{block.timestamp} will not exceed 64 bits for the next 500 billion years \bigskip
	\item \code{block.number}
\end{itemize}
\end{frame}

\begin{frame}{Future Plans - Effective Callback Freeness}
\begin{itemize}
	\item Recently introduced by~\cite{Grossman} \bigskip
	\item State variable invariants still hold after external calls
\end{itemize}
\end{frame}

\begin{frame}{Let's build together}
\url{github.com/ethereum/solidity}\\~\\
\url{gitter.im/ethereum/solidity-dev}\\~\\
\url{leo@ethereum.org}\\~\\
\begin{center}
Thank you!
\end{center}
\end{frame}

\begin{frame}[allowframebreaks]
  \frametitle<presentation>{References}
    
  \begin{thebibliography}{20}
    
  %\beamertemplatearticlebibitems
  \setbeamertemplate{bibliography item}{\insertbiblabel}

  \bibitem{Grossman}
Grossman, S., Abraham, I., Golan-Gueta, G., Michalevsky, Y., Rinetzky, N.,
  Sagiv, M., Zohar, Y.: Online detection of effectively callback free objects
  with applications to smart contracts. Proc. ACM Program. Lang.  2(POPL),
  48:1--48:28 (2017)

  \bibitem{Bhargavan2016}
Bhargavan, K., Delignat-Lavaud, A., Fournet, C., Gollamudi, A., Gonthier, G.,
  Kobeissi, N., Kulatova, N., Rastogi, A., Sibut-Pinote, T., Swamy, N.,
  Zanella-B{\'e}guelin, S.: Formal verification of smart contracts: Short
  paper. In: Proceedings of the 2016 ACM Workshop on Programming Languages and
  Analysis for Security. pp. 91--96. PLAS '16 (2016)

  \bibitem{Mythril}
ConsenSys: Mythril (2018), \url{github.com/ConsenSys/mythril}

  \bibitem{ZEUS}
Kalra, S., Goel, S., Dhawan, M., Sharma, S.: {ZEUS}: Analyzing safety of smart
  contracts  (2018)

  \bibitem{Luu2016}
Luu, L., Chu, D.H., Olickel, H., Saxena, P., Hobor, A.: Making smart contracts
  smarter. In: Proceedings of the 2016 ACM SIGSAC Conference on Computer and
  Communications Security. pp. 254--269. CCS '16 (2016)

  \bibitem{NKSSH18}
Nikolic, I., Kolluri, A., Sergey, I., Saxena, P., Hobor, A.: Finding the
  greedy, prodigal, and suicidal contracts at scale. CoRR  abs/1802.06038
  (2018), \url{http://arxiv.org/abs/1802.06038}

  \bibitem{Why3}
Why3: Why3 (2018), \url{why3.lri.fr}

  \bibitem{Eth-Isabelle}
Eth-Isabelle, \url{github.com/pirapira/eth-isabelle}

  \bibitem{KEVM}
KEVM, \url{github.com/kframework/evm-semantics}

  \bibitem{Ethereum-Lem}
Ethereum-Lem, \url{github.com/mrsmkl/ethereum-lem}

  \end{thebibliography}
\end{frame}

\end{document}


